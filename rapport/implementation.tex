%Différente classes qui intéragissent ensemble

Totoscope est un logiciel de rotoscopie utilisant la librairie Qt. Nous avons donc essayer d'utiliser les fonctionnalités de cette librairie au mieux.

Notre logiciel est composé de plusieurs fenêtres, nous allons tout d'abord décrire les choix opérés pour la fenêtre principale.

\subsection{Fenêtre d'ouverture}
	

\subsection{Fenêtre principale}
	Comme décrit dans notre premier rendu, nous avons décidé de rendre les fonctionnalités principales le plus accessible possible. 
	
	%\includegraphics[width=cm]{./figures/principale.png}

	Notre interface bien que simple et épurée, permet d'effectuer les actions essentielles à l'utilisation du logiciel.
	
	La fenêtre peut être divisée en cinq parties distinctes.
	\subsubsection{La barre de menu principale}
	Elle répertorie toutes les actions possibles de l'application. Elle est fractionnée en trois sous-menus :
	\begin{itemize}
		\item[Fichier] \hfill \\
			Ce menu contient des actions telles que créer un nouveau projet, ouvrir un projet déjà exitant, enregistrer ou exporter le travail courant ou encore fermer le projet et le logiciel.
		\item[\'Edition] \hfill \\
			Le menu \'Edition présente les actions fonctionnelles de l'application comme le fait d'annuler la dernière action, ou de la rétablir, l'activation ou non des pelures d'oignon et de la vidéo en arrière plan, et les actions concernant la lecture des dessins déjà effectués.
		\item[Aide] \hfill \\
			Ce menu quant à lui, ne contient que l'action A Propos qui donne accès aux informations concernant l'application.
	\end{itemize}
	
	\subsubsection{La barre de contrôle de la vidéo}
	
	\subsubsection{La barre latérale de dessin}
	
	\subsubsection{La zone de dessin}
	
	\subsubsection{Les miniatures}

\subsection{Fenêtre d'exportation}
	\subsubsection{Exportation au format images}
	
	\subsubsection{Exportation eu format vidéo}